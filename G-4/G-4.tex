\documentclass[12pt]{article}
\usepackage{amsmath}
\usepackage{amssymb}
\usepackage{listings}
\usepackage{framed}
\usepackage{graphicx}
\usepackage{algorithm}
\usepackage{algorithmicx}
\usepackage{algpseudocode}

\author{Sean White, Kierstyn Brandt, Rostik Mertz, Norman Tang}
\title{CSCI 432 - Assignment G4}

\begin{document}
\maketitle

\noindent
\textbf{Question A:} Solve the following recurrences: \smallskip

a) Show that $T(n) = T(n - 1) + n$ is $O(n^2)$ using the substitution method.\smallskip

To say that $T(n) = T(n - 1) + n$ is $O(n^2)$ is to say that $T(n)$ has an asymptotic upper bound of $n^2$. To show this is true we need to make a guess at a closed upper bound, so let that be $cn^2 - b$ where $c$ and $b$ are arbitrary constants. Now our relation is $T(n) \leq cn^2 -b$ so we see

\begin{align*}
T(n) &= T(n - 1) + n\\
&\leq c(n-1)^2 - b + (cn^2 - b)\\
&= c(n^2 - 2n + 1) + cn^2 -2b\\
&= 2cn^2 - 2cn - 2b + c\\
&= cn^2 - cn - 2b + c\\
&\leq cn^2 - b
\end{align*}

This is true $\forall c \geq b \geq 0$ so we set $c = 1$ and $b = 0$ which is allowed by the definition of big O and get that $T(n) \leq O(n^2)$.
\bigskip

b) Use a recursion tree to determine a good asymptotic upper bound for $T(n) = T(n/2) + n^2$. \smallskip

c) Use the master method to solve $T(n) = 2T(n/4) + 1$ \smallskip

d) Use the master method to solve $T(n) = 2T(n/4) + \sqrt{n}$ \smallskip

$T(n) = 2T(\frac{n}{4})+\sqrt[]{n}$, can otherwise be written as: $T(n) = 2T(\frac{n}{4})+n^{1/2}$. \\
We set our variables as: $a=2, b=4, f(n)=n^{1/2}$ and $\frac{1}{2}$, s.t. $a\geq1$\&$b>1$, and $f(n)$ is an asymptotically positive function on n. $\log_4 2 = \frac{1}{2} = c$, so this is a case 2 of the theorem. We then use the identity function: $f(n) = \theta(n^{log_ba}log^{k+1} n)$.
We plug in the values and get: $\sqrt{n}log^1n$. Hence, $T(n)= \theta\sqrt{n}log^1n$ \\

e) Use the master method to solve $T(n) = 2T(n/4) + n$ \smallskip

\noindent
\textbf{Question B:} Consider the randomized algorithm to compute the smallest enclosing ball. Consider the example set in the figure included on the assignment sheet. Suppose that the 'random' choice of vertex always chooses such that the vertex has the smallest index. What are the return values of each recursive call? \smallskip

\end{document}