\documentclass[12pt]{article}
\usepackage{amsmath}
\usepackage{amssymb}
\usepackage{listings}
\usepackage{framed}

\author{Sean White, Kierstyn Brandt, Rostik Mertz, Norman Tang}
\title{CSCI 432 - Assignment G2}

\begin{document}
\maketitle

\textbf{Question A.} Let A be an unsorted array of n integers, with A[0] ≥ A[1] and A[n−2] ≤ A[n − 1]. Call an index i a local minimum if A[i] is less than or equal to its neighbors.\\

(a) How would you efficiently find a local minimum, if one exists?\\

(b) For any loops in your algorithm, state what the loop invariant is.\\

\textbf{Question B.} The rand() function in the standard C library returns a uniformly random number in [0,RANDMAX-1]. Does rand() mod n generate a number uniformly distributed in [0, n−1]?\\

\textbf{Question C.} In this class, we assume the real-RAM model of computation for our analysis. Explain why we must define what model of computation we are using in an algorithms class, or, more generally, when talking about the complexity of an algorithm.\\

\textbf{Question D.} Which of the following correctly capture the runtime complexity of Mergesort. (Remember, you are expected to justify your answers to questions like this).\\

(a) O(n).\\

(b) O(n log n).\\

(c) O(n 2).\\

(d) Ω(n).\\

(e) Ω(n log n).\\

(f) Ω(n 2).\\

(g) Θ(n).\\

(h) Θ(n log n).\\

(i) Θ(n 2).\\

\end{document}