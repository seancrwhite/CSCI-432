\documentclass[12pt]{article}
\usepackage{amsmath}
\usepackage{amssymb}
\usepackage{listings}
\usepackage{framed}
\usepackage{graphicx}
\usepackage{algorithm}
\usepackage{algorithmicx}
\usepackage{algpseudocode}

\author{Sean White, Kierstyn Brandt, Rostik Mertz, Norman Tang}
\title{CSCI 432 - Assignment G5}

\begin{document}
\maketitle

\noindent
\textbf{Question A:} For the algorithm in 16.12 (Find the Longest Nondecreasing Subsequence), what is the loop invariant? Note: be sure to fully justify. \smallskip


\bigskip
\noindent
\textbf{Question B:} Add one source and one sink vertex ($s$ and $t$, respectively) to Figure 1. Connect $s$ to three vertices, and connect two different vertices to $t$. Add direction and weights to all edges so that the maximum flow / minimum cut of the flow network has value five (5). Annotate the edges clearly so that the max flow is shown, and draw a ‘blob’ (or, a topological disk) that shows what vertices are in S for the minimum cut. \smallskip

\bigskip
\noindent
\textbf{Question C:} Consider the graph in Figure 1. Answer the following questions regarding this graph.\smallskip

\smallskip
\noindent
(a) What is the Euler characteristic of the graph?\smallskip

\smallskip
\noindent
(b) What is the graph genus? (minimum number of handles necessary to draw the graph without crossing)?\smallskip

\smallskip
\noindent
(c) Is an Eulerian tour possible?\smallskip

\smallskip
\noindent
(d) Is a Hamiltonian path possible?\smallskip

\smallskip
\noindent
(e) What is the size of the largest clique?
\smallskip

\end{document}